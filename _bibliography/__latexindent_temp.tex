@article{DegenTG2015j,
abstract = {World knowledge enters into pragmatic utterance interpreta- tion in complex ways, and may be defeasible in light of speak- ers' utterances. Yet there is to date a surprising lack of sys- tematic investigation into the role of world knowledge in prag- matic inference. In this paper, we show that a state-of-the-art model of pragmatic interpretation greatly overestimates the in- fluence of world knowledge on the interpretation of utterances like Some of the marbles sank. We extend the model to cap- ture the idea that the listener is uncertain about the background knowledge the speaker is bringing to the conversation. This extension greatly improves model predictions of listeners' in- terpretation and also makes good qualitative predictions about listeners' judgments of how ‘normal' the world is in light of a speaker's statement. Theoretical and methodological implica- tions are discussed. Keywords:},
address = {Austin, TX},
author = {Degen, Judith and Tessler, Michael Henry and Goodman, Noah D},
editor = {Noelle, D. C. and Dale, R. and Warlaumont, A. S. and Yoshimi, J. and Matlock, T. and Jennings, C. D. and Maglio, P. P.},
journal = {Proceedings of the 37th Annual Conference of the Cognitive Science Society},
keywords = {computational pragmatics,experimental pragmatics,how often do you,if not always,in water,liefs,now imagine read-,prior be-,prior beliefs,probably extremely often,scalar implicature,think marbles would sink,world knowledge},
number = {2},
pages = {548--553},
publisher = {Cognitive Science Society},
title = {{Wonky worlds: Listeners revise world knowledge when utterances are odd}},
Website = {papers/2015_DegenTesslerGoodman.pdf},
year = {2015}
}
@inproceedings{DegenT2011,
author = {Degen, Judith and Tanenhaus, Michael K.},
booktitle = {Proceedings of the 33rd Annual Conference of the Cognitive Science Society},
editor = {Carlson, L. and H{\"{o}}lscher, C. and Shipley, T.},
keywords = {experimental pragmatics,eye-tracking,scalar implicature,subitizing},
pages = {3299--3304},
title = {{Making inferences: the case of scalar implicature processing}},
Website = {papers/2011_DegenTanenhaus.pdf},
year = {2011}
}
@article{Degen2013,
abstract = {In the face of underspecified utterances, listeners routinely and without much apparent effort make the right kinds of pragmatic inferences about a speaker's intended meaning. This dissertation investigates the processing of scalar impli- catures as a way of addressing how listeners perform this remarkable feat. In particular, the role of context in the processing of scalar implicatures from some to not all is explored. Contrary to the widely held assumption that scalar impli- catures are highly regularized, frequent, and relatively context-independent, this dissertation suggests that they are in fact relatively infrequent and highly context- dependent; both the robustness and the speed with which scalar implicatures from some to not all are computed are modulated by the probabilistic support that the implicature receives from multiple contextual cues. Scalar implicatures are found to be especially sensitive to the naturalness or expectedness of both scalar and non-scalar alternative utterances the speaker could have produced, but didn't. A novel contextualist account of scalar implicature processing that has roots in both constraint-based and information-theoretic accounts of language processing is proposed that provides a unifying explanation for a) the varying robustness of scalar implicatures across different contexts, b) the varying speed of scalar implica- tures across different contexts, and c) the speed and efficiency of communication.},
author = {Degen, Judith},
institution = {University of Rochester},
title = {{Alternatives in Pragmatic Reasoning}},
Website = {papers/2013_Degen_PhDThesis.pdf},
year = {2013}
}
@inproceedings{HawkinsSDG2015,
author = {Hawkins, Robert X D and Stuhlm, Andreas and Degen, Judith and Goodman, Noah D},
abstract = {What makes a question useful? What makes an answer appropriate? In this paper, we formulate a family of increasingly sophisticated models of question-answer behavior within the Rational Speech Act framework. We compare these models based on three different pieces of evidence: first, we demonstrate how our answerer models capture a classic effect in psycholinguistics showing that an answerer’s level of informativeness varies with the inferred questioner goal, while keeping the question constant. Second, we jointly test the questioner and answerer components of our model based on empirical evidence from a question-answer reasoning game. Third, we examine a special case of this game to further distinguish among the questioner models. We find that sophisticated pragmatic reasoning is needed to account for some of the data. People can use questions to provide cues to the answerer about their interest, and can select answers that are informative about inferred interests.},
booktitle = {Proceedings of the 37th Annual Conference of the Cognitive Science Society},
keywords = {answer,answerer behavior,answers,bayesian,in prompting a relevant,it suggests that the,language understanding,models,pragmatics,question itself is important,questions,work has focused on},
title = {{Why do you ask? Good questions provoke informative answers.}},
Website = {papers/2014_HawkinsDegenStuhlmuellerGoodman.pdf},
year = {2015}
}
@article{Degen2015,
author = {Degen, Judith},
abstract = {A prevalent, but to date untested, assumption about lexicalized scalar implicatures such as those from 'some' to 'not all', is that they fall into the class of GCIs and as such, constitute a homogeneous class of highly regularized and context-independent implicatures. This paper reports a test of this assumption in which linguistically untrained participants’ implicature strength judgments were collected for naturally occurring utterances containing the word 'some' in a large-scale corpus-based web study. The results indicate that implicature strength is highly variable and systematically dependent on features of the linguistic context such as the partitive, determiner strength, and discourse accessibility. These results call into question the GCI status of scalar implicatures from 'some' to 'not all' and demonstrate the usefulness of corpora and web-based methods for challenging received wisdom, enriching the empirical landscape, and informing theory in pragmatics.},
doi = {10.3765/sp.8.11},
issn = {1937-8912},
journal = {Semantics and Pragmatics},
number = {11},
pages = {1--55},
publisher = {Semantics and Pragmatics},
title = {{Investigating the distribution of 'some' (but not 'all') implicatures using corpora and web-based methods}},
url = {http://dx.doi.org/10.3765/sp.8.11},
volume = {8},
Website = {papers/2015_Degen_SP.pdf},
year = {2015}
}
@article{ScontrasDG2017,
annote = {doi: 10.1162/OPMI{\_}a{\_}00005},
abstract = {From English to Hungarian to Mokilese, speakers exhibit strong ordering preferences in multi-adjective strings: “the big blue box” sounds far more natural than “the blue big box.” We show that an adjective’s distance from the modified noun is predicted not by a rigid syntax, but by the adjective’s meaning: less subjective adjectives occur closer to the nouns they modify. This finding provides an example of a broad linguistic universal—adjective ordering preferences—emerging from general properties of cognition.},
author = {Scontras, Gregory and Degen, Judith and Goodman, Noah D},
doi = {10.1162/OPMI_a_00005},
journal = {Open Mind},
month = {jan},
pages = {1--14},
publisher = {MIT Press},
title = {{Subjectivity Predicts Adjective Ordering Preferences}},
url = {http://dx.doi.org/10.1162/OPMI{\_}a{\_}00005},
Website = {papers/2017_ScontrasDegenGoodman.pdf},
year = {2017}
}
@article{Degen2007,
author = {Degen, Judith},
journal = {Publications of the Institute of Cognitive Science, University of Osnabr{\{}{\"{u}}{\}}ck},
title = {{Processing Scalar Implicatures: What Role Does the Question of {\{}D{\}}efault Play for the Debate Between ({\{}N{\}}eo-){\{}G{\}}riceanism and {\{}R{\}}elevance {\{}T{\}}heory?}},
volume = {3-2007},
Website = {papers/2007_Degen_BScThesis.pdf},
year = {2007}
}
@article{FrankeD2016,
author = {Franke, Michael and Degen, Judith},
doi = {10.1371/journal.pone.0154854},
abstract = {Recent advances in probabilistic pragmatics have achieved considerable success in modeling speakers’ and listeners’ pragmatic reasoning as probabilistic inference. However, these models are usually applied to population-level data, and so implicitly suggest a homogeneous population without individual differences. Here we investigate potential individual differences in Theory-of-Mind related depth of pragmatic reasoning in so-called reference games that require drawing ad hoc Quantity implicatures of varying complexity. We show by Bayesian model comparison that a model that assumes a heterogenous population is a better predictor of our data, especially for comprehension. We discuss the implications for the treatment of individual differences in probabilistic models of language use.},
journal = {PLoS ONE},
number = {5},
pages = {1--25},
title = {{Reasoning in Reference Games : Individual- vs . Population-Level Probabilistic Modeling}},
volume = {11},
Website = {papers/2016_FrankeDegen.PDF},
year = {2016}
}
@article{YildirimDTJ2016,
abstract = {Linguistic meaning has long been recognized to be highly context-dependent. Quantifiers like many and some provide a particularly clear example of context-dependence. For example, the interpretation of quantifiers requires listeners to determine the relevant domain and scale. We focus on another type of context-dependence that quantifiers share with other lexical items: talker variability. Different talkers might use quantifiers with different interpretations in mind. We used a web-based crowdsourcing paradigm to study participants' expectations about the use of many and some based on recent exposure. We first established that the mapping of some and many onto quantities (candies in a bowl) is variable both within and between participants. We then examined whether and how listeners' expectations about quantifier use adapts with exposure to talkers who use quantifiers in different ways. The results demonstrate that listeners can adapt to talker-specific biases in both how often and with what intended meaning many and some are used.},
author = {Yildirim, Ilker and Degen, Judith and Tanenhaus, Michael K. and Jaeger, T. Florian},
doi = {10.1016/j.jml.2015.08.003},
issn = {0749596X},
journal = {Journal of Memory and Language},
keywords = {Adaptation,Pragmatics,Quantifiers,Semantics,Talker-specificity},
pages = {128--143},
publisher = {Elsevier Inc.},
title = {{Talker-specificity and adaptation in quantifier interpretation}},
url = {http://dx.doi.org/10.1016/j.jml.2015.08.003},
volume = {87},
Website = {papers/2016_YildirimDegenTanenhausJaeger.pdf},
year = {2016}
}
@article{DegenT2016,
abstract = {{\textcopyright} 2016 Cognitive Science Society, Inc.Two visual world experiments investigated the processing of the implicature associated with some using a "gumball paradigm." On each trial, participants saw an image of a gumball machine with an upper chamber with orange and blue gumballs and an empty lower chamber. Gumballs dropped to the lower chamber, creating a contrast between a partitioned set of gumballs of one color and an unpartitioned set of the other. Participants then evaluated spoken statements, such as "You got some of the blue gumballs." Experiment 1 investigated the time course of the pragmatic enrichment from some to not all when the only utterance alternatives available to refer to the different sets were some and all. In Experiment 2, the number terms two, three, four, and five were also included in the set of alternatives. Scalar implicatures were delayed relative to the interpretation of literal statements with all only when number terms were available. The results are interpreted as evidence for a constraint-based account of scalar implicature processing.},
author = {Degen, Judith and Tanenhaus, Michael K.},
doi = {10.1111/cogs.12227},
issn = {15516709},
journal = {Cognitive Science},
keywords = {Alternatives,Eye-tracking,Pragmatics,Quantifiers,Scalar implicature},
number = {1},
pages = {172--201},
title = {{Availability of Alternatives and the Processing of Scalar Implicatures: A Visual World Eye-Tracking Study}},
volume = {40},
Website = {papers/2016_DegenTanenhaus.pdf},
year = {2016}
}
@inproceedings{GrafDHG2016,
address = {Austin, TX},
abstract = {Nominal reference is very flexible—the same object may be called a dalmatian, a dog, or an animal when all are literally true. What accounts for the choices that speakers make in how they refer to objects? The addition of modifiers (e.g. big dog) has been extensively explored in the literature, but fewer studies have explored the choice of noun, including its level of abstraction. We collected freely produced referring expressions in a multi-player reference game experiment, where we manipulated the object’s context. We find that utterance choice is affected by the contextual informativeness of a description, its length and frequency, and the typicality of the object for that description. Finally, we show how these factors naturally enter into a formal model of production within the Rational Speech-Acts framework, and that the resulting model predicts our quantitative production data. },
author = {Graf, Caroline and Degen, Judith and Hawkins, Robert X D and Goodman, Noah D},
booktitle = {Proceedings of the 38th Annual Conference of the Cognitive Science Society},
editor = {Papafragou, A. and Grodner, D. and Mirman, D. and Trueswell, J.C.},
keywords = {intropsychling},
mendeley-tags = {intropsychling},
pages = {2261--2266},
publisher = {Cognitive Science Society},
title = {{Animal, dog, or dalmatian? Level of abstraction in nominal referring expressions}},
Website = {papers/2016_GrafDegenHawkinsGoodman.pdf},
year = {2016}
}
@article{DegenG2014,
author = {Degen, Judith and Goodman, Noah D},
abstract = {A rarely discussed but important issue in research on pragmatic inference is the choice of dependent measure for estimating the robustness of pragmatic inferences and their sensitivity to contextual manipulations. Here we present the results from three studies exploring the effect of contextual manipulations on scalar implicature. In all three studies we manipulate the salient question under discussion and the perceptual availability of relevant set sizes. The studies differ only in the dependent measure used: Exp. 1 uses truth judgements, Exp. 2 uses word probability ratings, and Exp. 3 uses a direct measure of sentence interpretation. We argue that the first two are effectively measures of production, and find they are sensitive to our contextual manipulations. In contrast the interpretation measure shows no effect of context. We argue that this methodologically troubling finding can be understood and predicted by using the framework of probabilistic pragmatics.},
journal = {Proceedings of the 36th Annual Conference of the Cognitive Science Society},
keywords = {pragmatics,psycholinguistics,scalar implicature},
pages = {397--402},
title = {{Lost your marbles? The puzzle of dependent measures in experimental pragmatics}},
Website = {papers/2014_DegenGoodman.pdf},
year = {2014}
}
@inproceedings{DegenF2012,
author = {Degen, Judith and Franke, Michael},
booktitle = {Proceedings of the 16th Workshop on the Semantics and Pragmatics of Dialogue},
editor = {Brown-Schmidt, Sarah and Ginzburg, Jonathan and Larsson, S.},
pages = {2 -- 11},
title = {{Optimal Reasoning About Referential Expressions}},
Website = {papers/2012_DegenFranke.pdf},
year = {2012}
}
@article{DegenT2015,
abstract = {Three experiments investigated the processing of the implicature associated with some using a "gumball paradigm." On each trial, participants saw an image of a gumball machine with an upper chamber with 13 gumballs and an empty lower chamber. Gumballs then dropped to the lower chamber and participants evaluated statements, such as "You got some of the gumballs." Experiment 1 established that some is less natural for reference to small sets (1, 2, and 3 of the 13 gumballs) and unpartitioned sets (all 13 gumballs) compared to intermediate sets (6-8). Partitive some of was less natural than simple some when used with the unpartitioned set. In Experiment 2, including exact number descriptions lowered naturalness ratings for some with small sets but not for intermediate size sets and the unpartitioned set. In Experiment 3, the naturalness ratings from Experiment 2 predicted response times. The results are interpreted as evidence for a Constraint-Based account of scalar implicature processing and against both two-stage, Literal-First models and pragmatic Default models.},
archivePrefix = {arXiv},
arxivId = {arXiv:1011.1669v3},
author = {Degen, Judith and Tanenhaus, Michael K.},
doi = {10.1111/cogs.12171},
eprint = {arXiv:1011.1669v3},
isbn = {9788578110796},
issn = {03640213},
journal = {Cognitive Science},
keywords = {Alternatives,Pragmatics,Quantifiers,Scalar implicature},
number = {4},
pages = {667--710},
pmid = {25265993},
title = {{Processing scalar implicature A constraint-based approach}},
volume = {39},
Website = {papers/2015_DegenTanenhaus.pdf},
year = {2015}
}
@article{Tonhauser,
author = {Tonhauser, Judith and Beaver, David I and Degen, Judith},
abstract = {Projective content is utterance content that a speaker may be taken to be committed to even when the expression associated with the content occurs embedded under an entailment-canceling operator (e.g., Chierchia & McConnell-Ginet, 1990). It has long been observed that projective content varies in how projective it is (e.g., Karttunen, 1971; Simons, 2001; Abusch, 2010), though preliminary experimental research has been able to confirm only some of the intuitions about projection variability (e.g., Smith & Hall, 2011; Xue & Onea, 2011). Given the sparse empirical evidence for projection variability, the first goal of this paper was to investigate projection variability for projective content associated with 19 expressions of American English. The second goal was to explore the hypothesis, called the Gradient Projection Principle, that content projects to the extent that it is not at-issue. The findings of two pairs of experiments provide robust empirical evidence for projection variability and for the Gradient Projection Principle. We show that many analyses of projection cannot account for the observed projection variability and discuss the implications of our finding that projective content varies in its at-issueness for an empirically adequate analysis of projection.},
Website = {https://academic.oup.com/jos/article/35/3/495/5042798?guestAccessKey=0fa933df-299f-439f-b622-b130e730c3b8},
year = {2018},
doi = {10.1093/jos/ffy007},
pages = {495--542},
volume = {35},
number = {3},
journal = {Journal of Semantics},
title = {{How projective is projective content? Gradience in projectivity and at-issueness}}
}
@inproceedings{Qing,
author = {Qing, Ciyang and Lassiter, Daniel and Degen, Judith},
abstract = {A common dependent measure used in visual-world eye-tracking experiments is the proportion of looks to a visually depicted object in a certain time window after the onset of the critical stimulus. When interpreting such data, a common assumption is that looks to the object reflect the listener’s belief that the object is the intended target referent. While this is intuitively plausible (at least for paradigms in which the task requires selecting a referent), relatively little is known about how exactly the proportion of looks to an object is related to a listener’s current belief about that object. Here, we test a simple, explicit linking hypothesis: the proportion of looks to an object reflects the probability that the listener assigns to the object being the target. To test this hypothesis, we supplement the eye-tracking data from Leffel, Xiang, and Kennedy (2016) with an offline incremental decision task to measure participants’ beliefs about the intended referent at various points in the unfolding sentence, and assess the extent to which these beliefs predict the eye-tracking data. The results suggest that the degree to which an object is believed to be the referent is only one factor that affects eye movements in referential tasks. Preliminary free production data we have collected for the scenes suggests that utterance expectations also play a role. We discuss methodological implications of these results for experimental linguistics.},
Website = {papers/2018_QingLassiterDegen.pdf},
year = {2018},
booktitle = {Proceedings of the 40th Annual Conference of the Cognitive Science Society},
keywords = {eye-tracking,gradable adjectives,imprecision,linking functions,pragmatics,semantics,vagueness,visual world},
title = {{What do eye movements in the visual world reflect? A case study from adjectives}}
}
@inproceedings{Hahn,
author = {Hahn, Michael and Degen, Judith and Goodman, Noah and Jurafsky, Dan and Futrell, Richard},
abstract = {Across languages, adjectives are subject to ordering restrictions. Recent research shows that these are predicted by adjective subjectivity, but the question remains open why this is the case. We first conduct a corpus study and not only replicate the subjectivity effect, but also find a previously undocumented effect of mutual information between adjectives and nouns. We then describe a rational model of adjective use in which listeners explicitly reason about judgments made by different speakers, formalizing the notion of subjectivity as agreement between speakers. We show that, once incremental processing is combined with memory limitations, our model predicts effects both of subjectivity and mutual information. We confirm the adequacy of our model by evaluating it on corpus data, finding that it correctly predicts ordering in unseen data with an accuracy of 96.2 %. This suggests that adjective ordering can be explained by general principles of human communication and language processing.},
Website = {papers/2018_HahnEtAl.pdf},
year = {2018},
booktitle = {Proceedings of the 40th Annual Conference of the Cognitive Science Society},
title = {{An Information-Theoretic Explanation of Adjective Ordering Preferences}}
}

@incollection{DegenTanenhaus2019,
author = {Degen, Judith and Tanenhaus, Michael K.},
booktitle = {Handbook of Experimental Semantics and Pragmatics},
chapter = {3},
editor = {Cummins, Chris and Katsos, Napoleon},
isbn = {9780198791768},
publisher = {Oxford University Press},
title = {{Constraint-based pragmatic processing}},
year = {2019},
doi = {10.1093/oxfordhb/9780198791768.013.8},
Website = {papers/2019_DegenTanenhaus_preprint},
abstract = {Processing language requires integrating information from multiple sources, including context, world knowledge, and the linguistic signal itself. How is this information integrated? A range of positions on the issue is possible, spanned by two extreme positions: extreme informational privilege -- certain types of information are processed earlier in online processing and weighted most heavily in the resulting utterance interpretation; and extreme parallelism -- all information is processed in parallel and weighted equally in the resulting interpretation. In reviewing the current empirical landscape on scalar implicature processing, the chapter argues for a constraint-based approach to pragmatic processing, which is closer in spirit to the parallelism account than the informational privilege account. The approach is also extended to other pragmatic phenomena.}
}

@article{ScontrasEtAl2019,
author={Scontras, Gregory and Degen, Judith and Goodman, Noah},
year = {2019},
journal = {Semantics and Pragmatics},
title = {On the grammatical source of adjective ordering preferences},
doi = {10.3765/sp},
Website = {papers/2019_ScontrasEtAl.pdf},
abstract = {Scontras et al. (2017) present experimental evidence demonstrating that the best predictor of adjective ordering preferences in the English noun phrase is the subjectivity of the property named by any given adjective: less subjective adjectives are preferred linearly closer to the nouns they modify. The current work builds on this empirical finding by proposing that the reason subjectivity predicts adjective ordering preferences has to do with the hierarchical structure of nominal modification. Adjectives that are linearly closer to the modified noun are often structurally closer, composing with the noun before adjectives that are farther away. Pressures from successful reference resolution dictate that less subjective, more useful adjectives contribute their meaning to the resulting nominal earlier, in an attempt to more effectively limit the reference search space.},
keywords = {adjective ordering, subjectivity, hierarchical structure, modification, reference resolution}
}

@misc{Jasbi,
author = {Jasbi, Masoud and Waldon, Brandon and Degen, Judith},
doi = {10.3389/fpsyg.2019.00189},      
file = {:Users/judithdegen/cogsci/papers{\_}misc/JasbiWaldonDegen{\_}submitted.pdf:pdf},
issn = {1664-1078},   
journal = {Frontiers in Psychology},      
pages = {189},     
title = {{Linking hypothesis and number of response options modulate inferred scalar implicature rate}},
url = {https://www.frontiersin.org/article/10.3389/fpsyg.2019.00189},       
volume = {10},      
year = {2019},
abstract = {The past 15 years have seen increasing experimental investigations of core pragmatic questions in the ever more active and lively field of experimental pragmatics. Within experimental pragmatics, many of the core questions have relied on the operationalization of the theoretical notion of `implicature rate'. Implicature rate based results have informed the work on acquisition, online processing, and scalar diversity, inter alia. Implicature rate has typically been quantified as the proportion of `pragmatic' judgments in two-alternative forced choice truth value judgment tasks. Despite its theoretical importance, this linking hypothesis from implicature rate to behavioral responses has never been extensively tested. Here we show that two factors dramatically affect the `implicature rate' inferred from truth value judgment tasks: a) the number of responses provided to participants; and b) the linking hypothesis about what constitutes a `pragmatic' judgment. We argue that it is time for the field of experimental pragmatics to engage more seriously with its foundational assumptions about how theoretical notions map onto behaviorally measurable quantities, and present a sketch of an alternative linking hypothesis that derives behavior in truth value judgment tasks from probabilistic utterance expectations.},
website = {papers/2019_JasbiWaldonDegen.pdf},
keywords = {scalar implicature, methodology, linking hypothesis, experimental pragmatics, truth value judgment task},
}

@article{Tonhauser-submitted,
author = {Tonhauser, Judith and Degen, Judith and de Marneffe, Marie-Catherine and Simons, Mandy},
abstract = {Evaluative adjective sentences are standardly analyzed as presupposing the prejacent: for instance, 'Kim wasn’t smart to watch the movie' is taken to presuppose that Kim watched the movie (e.g., Norrick 1978, Barker 2002, Oshima 2009, Kertz 2010). This paper argues against an analysis of the prejacent as a lexically-specified presupposition and proposes instead that the projectivity of the prejacent depends on the question addressed by the utterance of the evaluative adjective sentence (Beaver and Clark 2008, Simons et al. 2017, Beaver et al. 2017). Evidence for the proposed analysis comes from a study of the projectivity of the prejacent in naturally occurring evaluative adjective sentences and two experiments that show that the at-issueness and projectivity of the prejacent are sensitive to the question addressed by the utterance of the evaluative adjective sentence. The proposed analysis is also compared with that of Karttunen et al. 2014, according to which evaluative adjective sentences are systematically ambiguous.},
year = {2020},
volume = {5},
number = {1},
pages = {87},
journal = {Glossa: a journal of general linguistics },
doi = {10.5334/gjgl.701},
Website = {papers/2018_TonhauserEtAl_submitted.pdf},
title = {{Evaluative adjective sentences: A question-based analysis of projection}}
}

@article{Degen2019,
author = {Degen, Judith and Trotzke, Andreas and Scontras, Gregory and Wittenberg, Eva and Goodman, Noah D},
doi = {10.1016/j.pragma.2018.11.015},
file = {:Users/judithdegen/cogsci/papers{\_}misc/2019{\_}DegenTrotzkeEtAl.pdf:pdf},
issn = {0378-2166},
journal = {Journal of Pragmatics},
pages = {33--48},
publisher = {Elsevier Ltd},
title = {{Definitely, maybe: A new experimental paradigm for investigating the pragmatics of evidential devices across languages}},
volume = {140},
year = {2019},
abstract = {We present a new experimental paradigm for investigating lexical expressions that conveydifferent strengths of speaker commitment. Specifically, we compare different evidentialcontexts for using modal devices, epistemic discourse particles, and statements with noevidential markers at all, examining the extent to which listeners' interpretations ofcertain types of evidential words and their judgments about speaker commitment differ instrength. We also probe speakers' production preferences for these different devices undervarying evidential circumstances. The results of our experiments shed new light on distinctions and controversies that play a key role in the current theoretical literature on thesemantics and pragmatics of modals and discourse particles. Our paradigm thus contributes to a domain of experimental research on evidential expressions that is only just takingshape at the crossroads of theoretical semantics/pragmatics and psycholinguistics; weprovide a potential starting point for approaching theoretical debates on the nature ofmodal evidential expressions from an experimental and context-oriented perspective.},
Website = {papers/2019_DegenTrotzkeEtAl.pdf},
keywords = {discourse particles,english,evidentials,german,modals,psycholinguistics}
}

@article{DegenEtAl-submitted,
author = {Degen, Judith and Hawkins, Robert X D and Graf, Caroline and Kreiss, Elisa and Goodman, Noah D},
title = {{When redundancy is useful: A Bayesian approach to `overinformative' referring expressions}},
year = {2020},
journal = {Psychological Review},
abstract = {Referring is one of the most basic and prevalent uses of language. How do speakers choose from the wealth of referring expressions at their disposal? Rational theories of language use have come under attack for decades for not being able to account for the seemingly irrational overinformativeness ubiquitous in referring expressions. Here we present a novel production model of referring expressions within the Rational Speech Act framework that treats speakers as agents that rationally trade off cost and informativeness of utterances. Crucially, we relax the assumption that informativeness is computed with respect to a deterministic Boolean semantics, in favor of a non-deterministic continuous semantics. This innovation allows us to capture a large number of seemingly disparate phenomena within one unified framework: the basic asymmetry in speakers' propensity to overmodify with color rather than size; the increase in overmodification in complex scenes; the increase in overmodification with atypical features; and the increase in specificity in nominal reference as a function of typicality. These findings cast a new light on the production of referring expressions: rather than being wastefully overinformative, reference is usefully redundant.},
volume = {127},
issue = {4},
pages = {591–621},
doi = {10.1037/rev0000186},
Website = {papers/2020_DegenEtAl.pdf},

keywords = {language production, reference, overinformativeness, experimental pragmatics, Bayesian modeling}}

@incollection{DegenTonhauser_submitted,
author = {Degen, Judith and Tonhauser, Judith},
booktitle = {Open Handbook of Linguistic Data Management},
title = {{Managing web experiments for psycholinguistics: An example from experimental semantics/pragmatics}},
year = {to appear},
Website = {papers/2020_DegenTonhauser-submitted.pdf},
abstract = {The current DMUC reports on the organization of an experimental semantics/pragmatics project that investigated the extent to which variability in projective content’s projectivity is predicted by that content’s at-issueness (Tonhauser, Beaver, & Degen, 2018). The project included four web-based experiments in which participants adjusted sliding scales to provide projectivity and at-issueness ratings for close to 300 items. The workflow and best practices we report here are generalizable to any sufficiently similar web-based study, as well as to in-lab studies and dependent measures that differ substantially from the reported ones, e.g., online measures like eye movements. We also acknowledge that experimental semantics/pragmatics projects increasingly include a computational cognitive modeling component in addition to standard experimental data analysis. While the current DMUC does not include a cognitive model, we add some remarks on how to maintain development and documentation of cognitive models towards the end of this chapter. The following also describes best practices within the first author’s interActive Language Processing Lab at Stanford (ALPS) as of the time of publication. We close with some reflections on what we would change, were we to start again from scratch.}
}

@inproceedings{Kreiss2019UncertainES,
author = {Kreiss, Elisa and Franke, Michael and Degen, Judith},
abstract = {Transmission of information by means of language is a potentially lossy process. Especially adjunct information, such as the graded degree of evidence, is a piece of information that seems prima facie likely to be distorted by reproduction noise. To investigate this issue, we present the results of a two- step iterated narration study: first, we collected a corpus of 250 crime story reproductions that were produced in parallel reproduction chains of 5 generations in depth, for 5 different seed stories; a second separate large-scale experiment then targeted readers’ interpretation of these reproductions. Crucially, strength of evidence for the guilt of each story’s suspect(s) was manipulated in the initial seed stories. Across generations, readers’ guilt perceptions decreased when the evidence was originally strong, but remained stable when evidence was originally weak. Analysis of linguistic measures revealed that dissimilarity between a seed story and its reproduction, story length, and amount of hedging language affected the readers’ own guilt perception and the readers’ attribution of guilt perception to the author differently. The results provide evidence that evidential information indeed influences guilt perception in complex ways.},
Website = {papers/2019_KreissFrankeDegen},
year={2019},
booktitle = {Proceedings of the 41st Annual Conference of the Cognitive Science Society},
keywords = {experimental pragmatics; iterated narration; transmission chains; uncertain evidence},
title = {{Uncertain evidence statements and guilt perception in iterative reproductions of crime stories}},
}

@inproceedings{Schuster,
author = {Schuster, Sebastian and Degen, Judith},
abstract = {Speakers exhibit variability in their choice between uncertainty expressions such as might and probably. Recent work has found that listeners cope with such variability by updating their expectations about how a specific speaker uses uncertainty expressions when interacting with a single speaker. However, it is still unclear to what extent listeners form speaker-specific expectations for multiple speakers and to what extent listeners are adapting to a situation independent of the speakers. Here, we take a first step towards answering these questions. In Experiment 1, listeners formed speaker-specific expectations after being exposed to two speakers whose use of uncertainty expressions differed. In Experiment 2, listeners who were exposed to two speakers with identical use of uncertainty expressions formed considerably stronger expectations than in Experiment 1. This suggests that listeners form both speaker-specific and situation-specific expectations. We discuss the implications of these results for theories of adaptation.},
Website = {papers/2019_SchusterDegen},
year = {2019},
booktitle = {Proceedings of the 41st Annual Conference of the Cognitive Science Society},
keywords = {psycholinguistics; semantics; pragmatics; adaptation; uncertainty expressions},
title = {{Speaker-specific adaptation to variable use of uncertainty expressions}},
}

@article{Schuster-submitted,
author = {Schuster, Sebastian and Degen, Judith},
title = {{I know what you’re probably going to say: Listener adaptation to variable use of uncertainty expressions}},
abstract = {Pragmatic theories of utterance interpretation share the assumption that listeners reason about alternative utterances that a speaker could have produced, but didn't. For such reasoning to be successful, listeners must have precise expectations about a speaker's production choices. This is at odds with the considerable variability across speakers that exists at all levels of linguistic representation. This tension can be reconciled by listeners adapting to the statistics of individual speakers. While linguistic adaptation is increasingly widely attested, semantic/pragmatic adaptation is underexplored. Moreover, what kind of representations listeners update during semantic/pragmatic adaptation – estimates of the speaker's lexicon, or estimates of the speaker's utterance preferences – remains poorly understood. In this work, we investigate semantic/pragmatic adaptation in the domain of uncertainty expressions like might and probably. In a series of web-based experiments, we find 1) that listeners vary in their expectations about a generic speaker's use of uncertainty expressions; 2) that listeners rapidly update their expectations about the use of uncertainty expressions after brief exposure to a speaker with a specific usage of uncertainty expressions; and 3) that listeners' interpretations of uncertainty expressions change after being exposed to a specific speaker. We present a novel computational model of semantic/pragmatic adaptation based on Bayesian belief updating and show, through a series of model comparisons, that semantic/pragmatic adaptation is best captured by listeners updating their beliefs both about the speaker's lexicon and their utterance preferences. This work has implications for both semantic theories of uncertainty expressions and psycholinguistic theories of adaptation: it highlights the need for dynamic semantic representations and provides evidence against accounts that cast adaptation as simple low-level priming.},
Website = {papers/2020_SchusterDegen_submitted.pdf},
journal = {Cognition},
volume = {203},
year = {2020},
pages = {104285},
doi = {10.1016/j.cognition.2020.104285},
keywords = {adaptation; language comprehension; experimental pragmatics; Bayesian cognitive modeling; uncertainty expressions}
}

@inproceedings{Waldon,
author = {Waldon, Brandon and Degen, Judith},
abstract = {Truth Value Judgment Task experiments (TVJTs) are a common means of investigating pragmatic competence, particularly with regards to scalar inference. We present a novel quantitative linking function from pragmatic competence to participant behavior on TVJTs, based upon a Bayesian probabilistic model of linguistic production. Our model captures a range of observed phenomena on TVJTs, including intermediate responses on a non-binary scale, population and individual-level variation, participant endorsement of false utterances, and variation in response due to so-called scalar diversity.},
Website = {papers/2020_WaldonDegen},
year = {2020},
volume = {3},
article = {3},
pages={10--19},
number = {1},
booktitle = {Proceedings of the Society for Computation in Linguistics},
doi = {10.7275/sg32-aq30},
keywords = {psycholinguistics; truth value judgement task; pragmatics; linking function},
title = {{Modeling Behavior in Truth Value Judgment Task Experiments}},
}

@inproceedings{SchusterChenDegen,
author = {Schuster, Sebastian and Chen, Yuxing and Degen, Judith},
abstract = {Pragmatic inferences often subtly depend on the presence or absence of linguistic features. For example, the presence of a partitive construction (of the) increases the strength of a so-called scalar inference: listeners perceive the inference that Chris did not eat all of the cookies to be stronger after hearing “Chris ate some of the cookies” than after hearing the same utterance without a partitive, “Chris ate some cookies”. In this work, we explore to what extent neural network sentence encoders can learn to predict the strength of scalar inferences. We first show that an LSTM-based sentence encoder trained on an English dataset of human inference strength ratings is able to predict ratings with high accuracy (r = 0.78). We then probe the model’s behavior using manually constructed minimal sentence pairs and corpus data. We find that the model inferred previously established associations between linguistic features and inference strength, suggesting that the model learns to use linguistic features to predict pragmatic inferences.},
Website = {papers/2020_SchusterChenDegen.pdf},
year = {2020},
pages={5387--5403},
doi = {1910.14254},
booktitle = {Proceedings of the 58th Annual Meeting of the Association for Computational Linguistics},
keywords = {computational pragmatics; scalar implicature; neural networks},
title = {{Harnessing the linguistic signal to predict scalar inferences}},
}

@inproceedings{WaldonDegen2020b,
  title={Symmetric alternatives and semantic uncertainty modulate scalar inference},
  author={Waldon, Brandon and Degen, Judith},
  Website = {papers/2020_WaldonDegen_b.pdf},
  abstract = {Scalar inferences are commonly assumed to involve both literal semantic interpretation and social cognitive reasoning. However, the precise way to characterize listeners’ representation of context - including the space of possible utterance alternatives as well as the space of possible conventional meanings associated with linguistic forms - is a matter of ongoing debate. We report a partial replication of a scalar inference priming study by Rees and Bott (2018), introducing a novel baseline condition against which to compare behavior across different priming treatments. We also investigate the effect of raising participants’ awareness of communicatively stronger alternatives that explicitly encode an exhaustive meaning (e.g. some but not all with respect to some). Our results suggest that exhaustive alternatives (which are ‘symmetric’ to canonical alternatives) can modulate the availability and strength of scalar inferences, and that semantic uncertainty is an independent channel through which scalar inferences are modulated. We discuss implications for theories of pragmatic competence.},
  booktitle={Proceedings of the 42nd Annual Meeting of the Cognitive Science Society},
  keywords = {experimental pragmatics; implicature; priming; adaptation; computational pragmatics},
  volume={42},
  year={2020}
}

@inproceedings{KursatDegen2020,
  title={Probability and processing speed of scalar inferences is context-dependent},
  author={Kursat, Leyla and Degen, Judith},
  Website = {papers/2020_KursatDegen.pdf},
  abstract = {Studies addressing the question of whether scalar inferences generally incur a processing cost have yielded conflicting results. Constraint-based accounts, which seek to unify these conflicting results, make a prediction which we test here: the probability of an interpretation and the speed with which it is processed depends on the contextual support it receives. We manipulated contextual support for the scalar inference in two truth-value judgment experiments by manipulating a lexical feature (presence of partitive “of the”) and a pragmatic feature (the implicit Question Under Discussion). Participants’ responder type – whether their majority response was pragmatic (reflecting the inference) or literal (reflecting its absence) – was the main predictor of response times: pragmatic responses were faster than literal responses when generated by pragmatic responders; the reverse was true for literal responders. We interpret this as further evidence against costly inference accounts and in support of constraint-based accounts of pragmatic processing.},
  booktitle={Proceedings of the 42nd Annual Meeting of the Cognitive Science Society},
  keywords = {psycholinguistics; experimental pragmatics; scalar inference; Question Under Discussion},
  volume={42},
  year={2020}
}

@inproceedings{PortelanceDegenFrank2020,
  title={Predicting Age of Acquisition in Early Word Learning Using Recurrent Neural Networks},
  author={Portelance, Eva and Degen, Judith and Frank, Michael C.},
  Website = {papers/2020_PortelanceDegenFrank.pdf},
  abstract = {Vocabulary growth and syntactic development are known to be highly correlated in early child language. What determines when words are acquired and how can this help us understand what drives early language development? We train an LSTM language model, known to detect syntactic regularities that are relevant for predicting the difficulty of words, on child-directed speech. We use the average surprisal of words for the model, which encodes sequential predictability, as a predictor for the age of acquisition of words in early child language. We compare this predictor to word frequency and others and find that average surprisal is a good predictor for the age of acquisition of function words and predicates beyond frequency, but not for nouns. Our approach provides insight into what makes a good model of early word learning, especially for words whose meanings rely heavily on linguistic context.},
  booktitle={Proceedings of the 42nd Annual Meeting of the Cognitive Science Society},
  keywords = {Language model; recurrent neural network; LSTM; language acquisition; age of acquisition; child directed speech; word learning.},
  volume={42},
  year={2020}
}

@inproceedings{KreissDegen2020,
  title={Production Expectations Modulate Contrastive Inference},
  author={Kreiss, Elisa and Degen, Judith},
  Website = {papers/2020_KreissDegen.pdf},
  abstract = {Contrastive inferences, whereby a listener pragmatically infers a speaker’s referential intention of a partial referring expression like the yellow by reasoning about other objects in the context, are notoriously unstable. We report a production-centric model of interpretation couched within the Rational Speech Act framework. Adjective production probabilities a listener expects for objects in a context drive the size of contrastive inferences: the greater the asymmetry in expectation for a speaker to use a pre-nominal adjective for the target rather than for competitors, the greater the listener’s resulting target preference. Modifier production probabilities were collected (Exp. 1) and used to make predictions about comprehension in an incremental decision task (Exp. 2). The model’s interpretation predictions are supported by the data. This account has the potential to explain the fluctuating appearance of contrastive inferences and shifts the explanatory focus away from contrastive inference towards online interpretation of referring expressions more broadly.},
  booktitle={Proceedings of the 42nd Annual Meeting of the Cognitive Science Society},
  keywords = {contrastive inference; RSA; typicality; incremental processing},
  volume={42},
  year={2020}
}

@inproceedings{Waldon2020,
    title = {{Linguistic interpretation as inference under argument system uncertainty: the case of epistemic must}},
    author = {Waldon, Brandon},
    abstract = {Modern semantic analyses of epistemic language (incl. the modal must) can be characterized by the ‘credence assumption’: speakers have full certainty regarding the propositions that structure their epistemic states. Intuitively, however: a) speakers have graded, rather than categorical, commitment to these propositions, which are often never fully and explicitly articulated; b) listeners have higher- order uncertainty about this speaker uncertainty; c) must φ is utensed to communicate speaker commitment to some conclusion φ and to indicate speaker commitment to the premises that condition the conclusion. I explore the consequences of relaxing the credence assumption by extending the argument system semantic framework first proposed by Stone (1994) to a Bayesian probabilistic framework of modeling pragmatic interpretation (Goodman and Frank, 2016).},
    booktitle = "Proceedings of Probability and Meaning (PaM2020)",
    Website = {papers/2020_Waldon.pdf},
    year = {to appear}
}

@misc{Hahnetal2020,
	 title={Modeling word and morpheme order in natural language as an efficient tradeoff of memory and surprisal},
	 abstract = {Memory limitations are known to constrain language comprehension and production, and have beenargued to account for crosslinguistic word order regularities. However, a systematic assessment of therole of memory limitations in language structure has proven elusive, in part because it is hard to ex-tract precise large-scale quantitative generalizations about language from existing mechanistic modelsof memory use in sentence processing. We provide an architecture-independent information-theoreticformalization of memory limitations which enables a simple calculation of the memory efficiency oflanguages.  Our notion of memory efficiency is based  on the idea of amemory–surprisal tradeoff:  acertain level of average surprisal per word can only be achieved at the cost of storing some amount ofinformation about past context. Based on this notion of memory usage, we advance theEfficient TradeoffHypothesis: the order of elements in natural language is under pressure to enable favorable memory-surprisal tradeoffs. We derive that languages enable more efficient tradeoffs when they exhibitinforma-tion locality: when predictive information about an element is concentrated in its recent past. We provideempirical evidence from three test domains in support of the Efficient Tradeoff Hypothesis: a reanalysisof a miniature artificial language learning experiment, a large-scale study of word order in corpora of54 languages, and an analysis of morpheme order in two agglutinative languages. These results suggestthat principles of order in natural language can be explained via highly generic cognitively motivatedprinciples and lend support to efficiency-based models of the structure of human language.},
	 DOI={10.31234/osf.io/nusqz},
	 publisher={PsyArXiv},
	 author={Hahn, Michael and Degen, Judith and Futrell, Richard},
	 year={submitted},
	 Website = {papers/2020_HahnEtAl.pdf}
}
@inproceedings{Li2021,
  author = {Li, Elissa and Schuster, Sebastian and Degen, Judith},
  booktitle = {Proceedings of the Society for Computation in Linguistics: Vol. 4},
  title = {Predicting scalar inferences from ``or'' to ``not both'' using neural sentence encoders},
  year = {2021},
  website = {papers/2021_LiEtAl.pdf}
}

@inproceedings{Waldon2021,
    Author = {Waldon, Brandon},
    Booktitle = {Proceedings of the 56th Meeting of the Chicago Linguistic Society (CLS 56)},
    Title = {Epistemic must and might: evidence that argumentation is semantically encoded},
    Year = {to appear},
    website = {papers/2021_Waldon.pdf}
}

@inproceedings{WaldonDegen2021,
    author = {Waldon, Brandon and Degen, Judith},
    booktitle = {Proceedings of the Society for Computation in Linguistics},
    title = {Modeling cross-linguistic production of referring expressions},
    year = {2021}, 
    volume = {4},
    website = {papers/2021_WaldonDegen.pdf}
}

